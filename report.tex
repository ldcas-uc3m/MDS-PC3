\documentclass[12pt]{article}

\addtolength{\oddsidemargin}{-.585in}
\addtolength{\evensidemargin}{.585in}
\addtolength{\topmargin}{-.585in}
\addtolength{\textheight}{.760in}
\addtolength{\textwidth}{1in}

\usepackage[utf8]{inputenc}
\usepackage[table]{xcolor}
\usepackage{listings}
\usepackage{float} 
\usepackage{graphicx}
\usepackage{hyperref}
% \usepackage{svg}
\usepackage[backend=biber, style=ieee, isbn=false,sortcites, maxbibnames=5, minbibnames=1]{biblatex}

\definecolor{azulUC3M}{RGB}{0,0,102}

\bibliography{bibliography}

\begin{document}

% cover
\input{cover.tex}

\section{Image extraction}

After downloading Volatile 2.6.1 from the \href{https://github.com/volatilityfoundation/volatility/releases/tag/2.6.1}{source repository}, I copied the \texttt{i9100-CM\_3.0.64.zip} file into the \texttt{volatile/plugins/overlays/linux/} folder, and executed the following command:

\begin{lstlisting}
$ sudo python2 vol.py -f i9100-CM.bin --profile=Linuxi9100-CM_3_0_64ARM 
linux_recover_filesystem --dump-dir ./output
\end{lstlisting}

Producing the complete image of the device.


\section{Hash extraction}

In Android devices, the password is stored in the file \texttt{/data/system/password.key}, in this case producing the following (concatenated) hashes:
\begin{itemize}
    \item \textbf{SHA1:} \texttt{a66a4a34a78aec1a7058c8fa3bb3b0f1cc537dd0}
    \item \textbf{MD5:} \texttt{42f0f3f909f87d0706dcf139ab37f86e}
\end{itemize}

The salt for the hash can be located in the \texttt{/data/system/locksettings.db} database. We can access the database with an SQL viewer, for example:
\begin{lstlisting}
$ sqlite3 locksettings.db
\end{lstlisting}

By using a query we can obtain the salt:
\begin{lstlisting}
sqlite> SELECT * FROM locksettings WHERE name='lockscreen.password_salt';
\end{lstlisting}

Which returns the number \texttt{-6140990771726895285} (in decimal), which translates to \texttt{aac6d16df244374b} in hexadecimal (signed 2's complement).

\section{Password cracking}
I used Hashcat in order to crack this password. As we know the format of the password, we'll use a mask attack, assuming the "X"s can be any ascii character. As MD5 is faster than SHA1, I decided to use that format to crack it.

The attack would therefore be:
\begin{lstlisting}
$ hashcat -m 10 42f0f3f909f87d0706dcf139ab37f86e:aac6d16df244374b 
-a 3 'INS{?a?a?a?a?a}' -o out.out
\end{lstlisting}

The resulting cracked password was \texttt{INS\{t1MmY\}}.

\end{document}
